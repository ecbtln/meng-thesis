%% This is an example first chapter.  You should put chapter/appendix that you
%% write into a separate file, and add a line \include{yourfilename} to
%% main.tex, where `yourfilename.tex' is the name of the chapter/appendix file.
%% You can process specific files by typing their names in at the 
%% \files=
%% prompt when you run the file main.tex through LaTeX.
\chapter{Introduction to Docker}



\section{System Details}


\subsection{cgroups}

\subsection{LinuX Containers (LXC)}

% This is an example of how you would use tgrind to include an example
% of source code; it is commented out in this template since the code
% example file does not exist.  To use it, you need to remove the '%' on the
% beginning of the line, and insert your own information in the call.
%
%\tagrind[htbp]{code/pmn.s.tex}{Post Multiply Normalization}{opt:pmn}

As you can see, the intermediate results can be multiplied together, with no
need for intermediate normalizations due to the guard bit.  It is only at
the end of the operation that the normalization must be performed, in order
to get it into a format suitable for storing in memory\footnote{Note that
for purposed of clarity, the pipeline delays were considered to be 0, and
the branches were not delayed.}.

\subsection{A Union File System (AUFS)}

% This is an example of how you would use tgrind to include an example
% of source code; it is commented out in this template since the code
% example file does not exist.  To use it, you need to remove the '%' on the
% beginning of the line, and insert your own information in the call.
%
%\tgrind[htbp]{code/be.s.tex}{Block Exponent}{opt:be}

\section{Dockerfile}

\section{Orchestration}

\section{Industry Hype / Response}
\subsection{Comparison to Virtual Machines}
\subsection{Competitors}
\subsubsection{Flocker}
\subsubsection{Spoonium}

\section{Benefits \& Drawbacks}
https://news.ycombinator.com/item?id=8167928

\section{Docker Design Patterns}


