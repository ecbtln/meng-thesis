%% This is an example first chapter.  You should put chapter/appendix that you
%% write into a separate file, and add a line \include{yourfilename} to
%% main.tex, where `yourfilename.tex' is the name of the chapter/appendix file.
%% You can process specific files by typing their names in at the 
%% \files=
%% prompt when you run the file main.tex through LaTeX.
\chapter{Related Work}
\label{chap:relatedwork}
In this chapter, we discuss existing research in the fields relating to both container alternatives and conversion mechanisms. The existing work in this field is fairly limited, and to the extent of our research there is no existing tool that attempts to automate the conversion from VM to container. Everything must be done manually. Nevertheless, there are a few exist tools, as well as terminologies that are important in our overall discussion.

In sections~\ref{sec:dockerimport} and~\ref{sec:dockercommit}, we introduce two tools built into the Docker engine and describe why they are insufficient for the VM conversion process. Then, in section~\ref{sec:dockerize} we describe the term "dockerization" and how it is currently used in the industry. In section~\ref{sec:pvc} we mention Parallels Virtuozzo Containers and their connection to Docker. Finally, in section~\ref{sec:dtovm} we describe the alternative conversion, from Docker to Virtual Machine.

\section{Docker import}
\label{sec:dockerimport}
Included with Docker is the import tool that has the following documentation:\\

\texttt{Usage: docker import URL|- [REPOSITORY[:TAG]]}

\texttt{Create an empty filesystem image and import the contents of the tarball (.tar, .tar.gz, .tgz, .bzip, .tar.xz, .txz) into it, then optionally tag it.} \\

Theoretically, this tool would permit us to create a tarball out of a full-fledged virtual machine and import it directly into Docker. However, doing so would completely abandon the notion of layering different filesystems together to construct an image, and the final Docker image would take up the same space as the original virtual machine. Furthermore, if multiple virtual machines were converted in this manner, even if they had the same underlying OS and distribution, their layers wouldn't share any of the same lineage. Thus, two docker images would take up the same amount of space as the original two VMs. 

As a result, Docker recommends that this tool be used only to create base images. Base images should be made as small as possible and intuitively should only contain the necessary files to provide a fully-functioning OS. Packages can subsequently be installed on top of the base image depending on the desired function of the container. 

\section{Docker commit}
\label{sec:dockercommit}
Also, included with Docker is the import tool that has the following documentation:\\

\texttt{Usage: docker commit [OPTIONS] CONTAINER [REPOSITORY[:TAG]]}

\texttt{Create a new image from a container's changes} \\

As previously mentioned, a container represents the top-most writable layer of the filesystem, while images are the read-only layers from which a container can inherit. In the course of a container's lifetime, as changes are accrued, one can optionally choose to save these changes to a read-only layer, to be used in future images or containers, by using the \texttt{docker commit} tool.

\section{"Dockerization"}
\label{sec:dockerize}
Dockerization is the process of converting an application to be capable of being run within the Docker framework. Existing containers that run nginx, apache, or mysql, for example, are said to have been "Dockerized" from their original forms. This process generally consists of someone picking a framework or tool for which there is no publicly available Docker image. Then a Dockerfile can be generated that provides instructions on how to build the corresponding environment for the software. Finally, the Dockerfile and any associated files are shared to the public Docker registry, allowing other users to download and build these images for private use. The mapping from application to container is by no means automatic and generally one must manually write the instructions in the Dockerfile. 

Of particular importance is the distinction between making use of \texttt{docker commit}, as mentioned in section~\ref{sec:dockercommit}, as compared to creating a Docker image that is built purely from a set of instructions in a Dockerfile. Docker commit is not accepted as a valid Dockerization tool because it masks the exact changes made to the filesystem layer and generally results in Docker images that take up more space than if one were to build the same image from a set of instructions. 

In fact, Docker has recently started labeling "trusted builds" as those in the registry for which there is a complete set of instructions of how they are built in a publicly accessible GitHub repository. The builds are "trusted" in the sense that the complete set of instructions of how each image is built is observable by all potential users. 

Thus, VM2Docker aims to streamline the Dockerization process by offering the automatic and large-scale conversion of VMs to their corresponding Docker containers. We specifically focus on the filesystem layering process for evaluation and we do not make use of the ~\texttt{docker commit} tool in favor of greater transparency of the set of instructions used to build each image.

\section{Parallels Virtuozzo Containers (PVC)}
\label{sec:pvc}
http://kb.sp.parallels.com/en/6324
%http://download.swsoft.com/pva/60/docs/en/html/Parallels_Virtual_Automation_Administrators_Guide/03-02-01-31-02.htm
Converted - %{http://www.theregister.co.uk/2014/12/15/parallels_to_adopt_docker_as_native_app_format_in_cloud_server/

%\section{Docker Desktop}
%https://blog.docker.com/2013/07/docker-desktop-your-desktop-over-ssh-running-inside-of-a-docker-container/

\section{Docker $\to$ VM}
\label{sec:dtovm}
Unlike the task of converting VMs to Docker, the converse is extremely straightforward. Since Docker can be run within a VM, one can simply run the same Docker container within a VM to obtain a Docker image that has been ``converted" to a VM.