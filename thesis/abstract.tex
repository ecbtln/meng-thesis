% $Log: abstract.tex,v $
% Revision 1.1  93/05/14  14:56:25  starflt
% Initial revision
% 
% Revision 1.1  90/05/04  10:41:01  lwvanels
% Initial revision
% 
%
%% The text of your abstract and nothing else (other than comments) goes here.
%% It will be single-spaced and the rest of the text that is supposed to go on
%% the abstract page will be generated by the abstractpage environment.  This
%% file should be \input (not \include 'd) from cover.tex.
Container technology represents a flourishing field in cloud computing. For many types of computing, containers are a viable alternative to virtual machines because they are more lightweight and higher performing than the alternative. Containers share the kernel with the host, as opposed to virtual machines which have a completely isolated kernel. This distinction is responsible for the speed improvements of containers, but also results in less isolation and therefore increased security concerns. The Docker framework, among other alternatives, has gotten the most attention and popularity over the past year and provides a powerful layered filesystem to improve deployability and space savings for those containers that share many layers in common. Given the recent trend towards container technology, many types of VMs might be more suitable to be run within a container. As of this writing, there is no system for automatically converting VMs to containers, as all configuration must be done manually. This is potentially unwieldy for system administrators looking to convert five to ten, to hundreds, of virtual machines at once.  This thesis presents a system we call VM2Docker that attempts to automate this conversion for many virtual machines at once. VM2Docker specifically focuses on automatically generating layers for Docker to take advantage of the filesystem similarities across VMs of the same operating system. VM2Docker has been tested on various releases of Ubuntu, CentOS, and Mageia with a large degree of success and is able to provide filesystem space savings and deployment speed improvements with as few as 2 instances of a VM of a given operating system and release.
